\documentclass[a4paper,12pt]{report}

\usepackage[T2A]{fontenc}
\usepackage[utf8]{inputenc}
\usepackage[english,ukrainian]{babel}
\usepackage{amsmath}
\usepackage{ragged2e}   
\usepackage{listings}
\usepackage{hyperref}

\usepackage{titlesec}
\titleformat{\section}[block]{\Large\bfseries\filcenter}{}{1em}{}
\titleformat{\subsection}[block]{\bfseries\filcenter}{}{1em}{}

\author{Байбула Кирило Аленович}
\title{ПРОГРАМНА МОДЕЛЬ РОБОТИ ПРОЦЕСОРА}

\begin{document}

\begin{titlepage}
	\begin{center}
		\Large
		\textbf{Київський національний університет імені Т.Шевченка}
		\vspace{5cm}

		\Huge
		\textbf{Звіт}

		\LARGE
		До лабораторної роботи №3
		\vspace{0.5cm}

		\textbf{ПРОГРАМНА МОДЕЛЬ РОБОТИ ПРОЦЕСОРА}
		\vfill
	\end{center}

	\begin{FlushRight}
		Байбула Кирило Аленович \\
		Групa К-21 \\
		Факультету комп’ютерних наук \\
		та кібернетики
	\end{FlushRight}

	\vspace{0.5cm}

	\begin{center}
		\textbf{Київ} \\
		2021
	\end{center}

\end{titlepage}
\clearpage


\section{МЕТА}
Потрібно розробити програмну модель роботи співпроцесора з 8 регістрів,
об'єднаних у стек. Дані в регістрах подаються у 754 форматі з різною
довжиною характеристики та мантиси (задається у програмі, перед компіляцією).
Студенти отримують дрібно-раціональну функцію від двох змінних. Необхідно
написати послідовність дій для обчислення значення для заданих значень
змінних. Далі ця послідовність реалізується у межах програми 
(обробляється послідовність операторів обчислення функції, враховують
станкову організацію регістрів).

\section{ХІД РОБОТИ}
Для простоти написання були використані мова програмування \textbf{Пайтон}
та \textbf{ООП} підхід до написання програми. Де до головних сутностей можна віднести
стек, регістри та дробові числа, що хранить стек. Через методи були реалізовані
базові опертори над стеком як \verb|pop|, \verb|push|, \verb|add| та \verb|div|,
що прибирають та додають елементи у стек, знаходять суму двох верхніх значень
та ділять два верхній значення відповідно. Також за варіантом мені потрібно
було обчислити вираз:
     \[(a+b)/b\]

Приклад запуску програми для значень \verb|a = 1.2|, \verb|b = 3.2|:
\begin{verbatim}
    push a
    R1: 00111111 10011001 10011001 10011010

    push b
    R2: 01000000 01001100 11001100 11001101
    R1: 00111111 10011001 10011001 10011010

    add
    R1: 01000000 10001100 11001100 11001101

    push b
    R2: 01000000 01001100 11001100 11001101
    R1: 01000000 10001100 11001100 11001101

    div
    R1: 00111111 10110000 00000000 00000000
\end{verbatim}

Як бачимо для обчислення нам потрібно спочатку покласти значення a і b у стек
занйти їх суму, покласти зачення b та зробити ділення.

\section{ВИСНОВОК}
У ході цієї роботи я розробив імітаційну модель сопроцесора, що працює з числами
з плаваючою крапкою у \textbf{IEEE} 754 стандарті.

\subsection{Посилання}
\begin{itemize}
    \item{Код лабораторної: \url{https://github.com/Velnbur/FNCP}}
\end{itemize}

\end{document}
